%\section{Conclusions}
\label{sec:cap6}

This work presented a comprehensive investigation into the reliability of SRAM-based FPGAs subjected to fault-prone environments, with an emphasis on experimental fault injection and dynamic fault recovery. A benchmark circuit was implemented and evaluated under controlled fault injection campaigns, enabling the assessment of fault susceptibility, detection efficiency, and recovery mechanisms. The experimental results demonstrated that integrating redundancy-based fault masking with runtime reconfiguration strategies can significantly improve system availability and fault coverage without requiring full system shutdown.

The proposed methodology proved effective for systematically evaluating FPGA reliability, offering a reproducible approach for both academic research and industrial validation. Moreover, the study confirmed that partial dynamic reconfiguration is a viable solution for mitigating the effects of faults in critical modules while minimizing downtime and resource overhead.

While the results were promising, the research also revealed limitations regarding reconfiguration latency, hardware resource usage, and the scalability of the proposed approach to more complex systems. Future work could explore adaptive redundancy schemes, machine-learning-based fault prediction, and the application of the proposed methodology to heterogeneous FPGA-SoC platforms.

Ultimately, this dissertation contributes to advancing the understanding of fault behavior in SRAM-based FPGAs and demonstrates practical pathways to improving their reliability in safety-critical and high-availability applications.

\section{Future works}
Although this work has demonstrated the feasibility and effectiveness of combining N-Modular Redundancy (NMR), Dynamic Partial Reconfiguration (DPR), and fault injection to enhance the reliability of SRAM-based FPGAs, several research directions remain open for further exploration:

\begin{itemize}
    \item \textbf{Integration of Error Correction Codes (ECC):}
    \begin{itemize}
        \item \textbf{To Block RAM (BRAM):} Incorporating ECC mechanisms into on-chip memories can significantly reduce the impact of soft errors on stored data. Implementing and testing single-error correction/double-error detection (SEC-DED) codes in BRAM would provide an additional layer of fault tolerance and complement the redundancy and reconfiguration strategies.
        \item \textbf{To Communication Buses:} Applying ECC to internal and external communication interfaces would help protect data transfers against transient faults, improving overall system integrity. Evaluating the trade-off between added latency, resource consumption, and error coverage would be a key aspect of this study.
    \end{itemize}

    \item \textbf{Mitigation Strategies on the Processing System (PS) Side:}
    Extending fault-tolerant techniques to the embedded processing system (e.g., ARM cores in \gls{soc} \gls{fpga} could enhance end-to-end reliability. Potential approaches include software-level redundancy, checkpoint/restart mechanisms, or selective hardening of critical software components.

    \item \textbf{Evaluation Across Multiple \gls{fpga} Vendors:}
    While this research has targeted a Xilinx Series 7 Zynq device, reliability characteristics, reconfiguration capabilities, and resource utilization vary across vendors and others device families. A comparative study using devices from different manufacturers would help generalize the proposed methodology, providing broader insights into portability and performance trade-offs.

    \item \textbf{Scalability and Resource Optimization:}
    As \gls{nmr} and \gls{dpr} techniques introduce additional hardware overhead, future work could explore automated design-space exploration to optimize the balance between fault coverage, resource consumption, and system performance.

    \item \textbf{In-Field Deployment and Long-Term Testing:}
    Extending the methodology to real operational environments, such as radiation test facilities or mission-critical field deployments, would allow validation of laboratory results under practical conditions.
\end{itemize}

\section{Publications}

\begin{itemize}
 \item ZANONE R.; SAOTOME O. \href{https://lcv.fee.unicamp.br/wp-content/images/BTSym23-Brasil/papers/BTSym2023_066.pdf}{A Base FPGA Platform for Safer Designs exposed to harsh radioactive environments.} In: \textbf{Brazilian Technology Symposium}. Campinas: BTSym 2023.
\end{itemize}

