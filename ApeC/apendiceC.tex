%\section{SEM IP Logic Utilization}
\label{apeC}

Figure~\ref{fig:semip-logic} shows the overall logic utilization of the \gls{semip} in the final design, highlighting the proportion of FPGA resources occupied by the configuration memory scrubbing and monitoring module.

\begin{figure}[H]
    \centering
    \includegraphics[width=0.5\linewidth]{ApeC/sem_ip_logic.png}
    \caption{Overall logic utilization of the \acrshort{semip} in the final design, showing the amount of FPGA resources occupied, such as \acrshort{lut}, \acrshort{ff}, and internal control blocks.}
    \label{fig:semip-logic}
\end{figure}

Figure~\ref{fig:semip-logic2} provides a detailed breakdown of the logic utilization of the \gls{semip}, separated by slice type and \gls{lut} element. It can be observed that the majority of resources correspond to combinational logic, which is characteristic of control and error-correction functions rather than sequential storage elements.

\begin{figure}[H]
    \centering
    \includegraphics[width=0.5\linewidth]{ApeC/sem_ip_logic2.png}
    \caption{Detailed breakdown of the \acrshort{semip} logic utilization by slice type and \acrshort{lut} element.}
    \label{fig:semip-logic2}
\end{figure}

Finally, Figure~\ref{fig:semip-specific} presents the specific \gls{fpga} primitives required by the \gls{semip}. These include essential building blocks such as FRAME\textunderscore ECCE2 and ICAPE2.

\begin{figure}[H]
    \centering
    \includegraphics[width=0.5\linewidth]{ApeC/sem_ip_specific.png}
    \caption{Specific \acrshort{fpga} primitives used by the \acrshort{semip}.}
    \label{fig:semip-specific}
\end{figure}

Figure~\ref{fig:semip-implementation} presents the implementation of \gls{semip} in \gls{fpga}. It is highlighted in yellow green, inside the red circle.

\begin{figure}[H]
    \centering
    \includegraphics[width=1\linewidth]{ApeC/core012&sem-impl-edited.png}
    \caption{Area where \acrshort{semip} is implemented inside the \acrshort{fpga}.}
    \label{fig:semip-implementation}
\end{figure}
