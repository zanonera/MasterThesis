%\section{Introduction}
\label{sec:cap1}

\glspl{fpga} are widely employed in low- and medium-volume electronic systems due to their flexibility, high performance, 
and short development cycles. As discussed by \textcite{Garcia2020}, these devices comprise a large array of configurable logic blocks and programmable interconnections, allowing customization for a wide range of applications and straightforward reprogramming of their configuration.

As discussed in \textcite{Battezzati2010,Garcia2020,Koers2024}, the most common \gls{fpga} stores its configuration in an external \gls{spi} flash memory that is read during boot into a \gls{sram}-based memory, called \gls{cram}. However, \gls{sram} are inherently susceptible to transient faults caused by radiation \cite{acmeTuto}, electromagnetic interference, or aging effects, which can corrupt the bits stored in the \gls{cram} memory and consequently lead to unexpected behavior, performance degradation, or even complete system failure \cite{Wubs2023}.

To allow the use of \gls{fpga} in applications where it will be subjected to harsh environments, as in space, two options are available. The use of space-grade components, that are engineered for radiation tolerance \cite{Battezzati2010,bates2016,DiMascio2019,Garg2024} or \gls{cots} \glspl{fpga} that are not engineered to withstand harsh environments, but could offer advantages in cost and performance \cite{Barbosa2021,Adria2023,Rogenmoser2023_HMR}. Advances in error mitigation techniques, such as redundancy and error correction, have made it possible to consider COTS devices for space applications \cite{10207287,Brosser2014}. 

\gls{sram}-based \glspl{fpga} offer an advantage through their capability to be dynamically and partially reconfigured at run time, making them considerably more powerful and flexible than non-dynamically reconfigurable technologies such as flash- or antifuse-based \glspl{fpga} \cite{Bernardeschi2015}.

\gls{cots} \glspl{fpga} are often preferred over space-grade \glspl{fpga} in modern aerospace applications, mainly, for two reasons. Firstly, \gls{cots} \glspl{fpga} offer significantly lower costs, making them more accessible for a wider range of missions, especially in the context of small satellites and constellations \cite{Yaman2022}, where budget constraints are critical. As presented by \textcite{Barbosa2021}, one radiation hardened \gls{fpga} could cost forty four (44) times more than the \gls{cots} alternative. Secondly, \gls{cots} \glspl{fpga} benefit from the rapid advances of the commercial electronics market, providing higher performance, higher logic density, and more up-to-date features compared to their space-grade counterparts, which typically lag behind in the technology nodes due to extended qualification processes \cite{Nguyen2017}.

A third reason is that many radiation-hardened devices face export restrictions, as their commercialization and distribution are controlled by \gls{itar} \cite{Nguyen2017}.

As presented by \textcite{9298818, Adria2023,Rogenmoser2023_HMR,paikowsky2017,Garcia2020}, the \emph{New Space} paradigm is based on the use of \gls{cots} components to lower costs and allow faster development cycles in software applications. \gls{cots} \glspl{fpga} are easier and faster to acquire, reducing lead times and facilitating rapid prototyping and testing. Although space-grade FPGAs offer improved radiation tolerance and reliability by design, advances in fault mitigation techniques have made it possible to use \gls{cots} devices effectively in many space environments. As discussed in \textcite{Barbosa2021}, \gls{esa} highlights the advantages of \gls{cots} \glspl{fpga}, especially in high-performance computation. 

Traditional fault-tolerant techniques, such as static redundancy or periodic device reprogramming, are often insufficient or inefficient in these scenarios. They may incur high area overhead, disrupt system operation during recovery, or fail to respond rapidly to localized failures. \gls{nmr} provides a robust mechanism for masking errors at run-time through majority voting, but its static nature cannot repair defective modules, leading to gradual degradation of resources.

\gls{dpr} and memory scrubbing emerges as a powerful complement to redundancy techniques by allowing the selective repair of faulty modules without interrupting system functionality. When combined with systematic fault injection, researchers can accurately evaluate the resilience of FPGA-based systems and validate mitigation strategies under realistic conditions. 

In addition, to address the high demand for computing power, while maintaining high design flexibility, \gls{soc} solutions have gained significant traction. \glspl{soc} integrate, within a single component, a \gls{fpga} fabric, one or more processor cores, multiple communication buses, and a wide variety of peripherals for data input and output. This integration enables efficient implementation of complex reconfigurable systems that can be upgraded or repurposed even after deployment in the field.

%%%%%%%%%%%%%%%%%%%%%%%%%%%%%%%%%%%%%%%%%%%%%%%%%%%%%%%%%%%%%%%%%%%%%%%%%%%
\section{Motivation}

Due to the increasing use of \gls{cots} \glspl{fpga} and \glspl{soc} in mission-critical and safety-critical applications, especially in the aerospace \cite{9298818}, it is important to apply error mitigation strategies to increase reliability, especially in unmanned devices. While static \gls{nmr} can mask faults, periodic full, \gls{dpr} and scrubbing can repair them. These approaches achieve better results when applied in combination, increasing tolerance to error, and achieving some degree of self-healing \cite{6880163,Shaji2019}.

Transient faults caused by radiation, interference, or aging can corrupt the \gls{cram} or logic fabric, leading to degradation in performance or complete failure. This work is motivated by the need for a unified approach that combines \gls{nmr} for fault masking, \gls{dpr} and scrubbing for runtime repair, and systematic fault injection for validation, enabling FPGA-based systems to tolerate and recover from errors adaptively while meeting the performance demands of next-generation critical applications.

Inspired by the framework developed by \textcite{8541452}, this study advances their reconfigurable platform by adding robust error mitigation, fault injection, and analysis capabilities.

By integrating \gls{nmr}, \gls{dpr}, memory scrubbing and \gls{cram} error injection into a unified study, this thesis aims to demonstrate a practical methodology for improving system availability, reducing downtime, and extending the operational lifetime of \gls{sram} based \glspl{fpga}, integrated in dual core processor \gls{soc}, in safety-critical environments.

%%%%%%%%%%%%%%%%%%%%%%%%%%%%%%%%%%%%%%%%%%%%%%%%%%%%%%%%%%%%%%%%%%%%%%%%%%%
\section{Scope and Objectives}

This section defines the scope and primary objectives of the research, describing the boundaries of the study and the specific goals pursued throughout its development. It establishes the foundation for the methodologies and contributions presented in the following chapters.

\subsection{Scope}
This work focuses on improving the reliability of \gls{sram}-based \glspl{fpga} and \glspl{soc} deployed in mission- and safety-critical applications. The research explores the integration of \gls{nmr} for fault masking with \gls{dpr} and memory scrubbing for runtime repair. In addition, a systematic fault injection framework is developed to validate the resilience of the proposed mitigation strategies under realistic fault conditions.

\subsection{Objectives}
The main objectives of this work are:
\begin{itemize}
    \item Develop and implement a \gls{nmr} architecture capable of masking faults in real time.
    \item Integrate \gls{dpr} techniques to selectively repair faulty modules without interrupting system operation.
    \item Apply memory scrubbing to maintain the integrity of the configuration data in \gls{fpga} \gls{cram}.
    \item Design and execute a fault injection campaign to evaluate system reliability and fault coverage.
    \item Assess the performance, resource utilization, and fault tolerance trade-offs of the proposed solution on \gls{soc} platforms.
\end{itemize}

%%%%%%%%%%%%%%%%%%%%%%%%%%%%%%%%%%%%%%%%%%%%%%%%%%%%%%%%%%%%%%%%%%%%%%%%%%%
\section{Work Contribution}
The main contributions of this work can be summarized as follows:
    \begin{itemize}
        \item {A fault-tolerant FPGA design methodology integrating NMR, DPR, and fault injection for reliability evaluation;}
        \item {A practical test framework using a benchmark circuit (e.g., a modular RISC-V processor core) to validate the approach under realistic fault scenarios;}
        \item {Experimental data and analysis quantifying the trade-offs between redundancy overhead, reconfiguration time, and system availability;}
        \item {Design guidelines for developing resilient FPGA architectures in mission- and safety-critical applications.}
    \end{itemize}

%%%%%%%%%%%%%%%%%%%%%%%%%%%%%%%%%%%%%%%%%%%%%%%%%%%%%%%%%%%%%%%%%%%%%%%%%%%
\section{Text Organization}
This dissertation is structured into six chapters, organized to guide the reader from fundamental concepts to experimental results and conclusions:

\begin{itemize}
    \item \hyperref[sec:cap1]{\textbf{Chapter 1 – Introduction:}} Presents the background, motivation, scope, and objectives of the work. Introduces \glspl{fpga} and \glspl{soc}, highlights their susceptibility to transient faults, and outlines the importance of combining \gls{nmr}, \gls{dpr}, and fault injection techniques to enhance the reliability of the system.
    
    \item \hyperref[sec:cap2]{\textbf{Chapter 2 – Concepts and Literature Review:}} Reviews the state-of-the-art in fault-tolerant FPGA design, including redundancy schemes, memory scrubbing, partial dynamic reconfiguration, and fault injection methodologies. This chapter provides a foundation for understanding the theoretical and practical aspects of reliability in SRAM-based FPGA systems.
    
    \item \hyperref[sec:cap3]{\textbf{Chapter 3 – Theoretical Introduction:}} This section provides the theoretical background that supports the proposed approach, addressing key topics such as \gls{nmr} principles, fault modeling, reconfiguration techniques, and reliability assessment metrics.
    
    \item \hyperref[sec:cap4]{\textbf{Chapter 4 – Development Methodology:}} Describes the design and implementation methodology of the proposed system, including benchmark selection, design of the FPGA architecture, integration of \gls{nmr} and \gls{dpr}, and setup of the fault injection framework.
    
    \item \hyperref[sec:cap5]{\textbf{Chapter 5 – Results and Analysis:}} Presents the experimental results obtained from the fault injection campaigns and the runtime reconfiguration tests. The chapter includes analysis of fault coverage, system performance, resource utilization, and the effectiveness of proposed mitigation strategies.
    
    \item \hyperref[sec:cap6]{\textbf{Chapter 6 – Conclusions:}} Summarizes the main findings of the dissertation, discusses contributions to the field, identifies limitations, and proposes directions for future work.
\end{itemize}

