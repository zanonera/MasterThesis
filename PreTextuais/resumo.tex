A confiabilidade de sistemas digitais baseados em FPGAs é um aspecto crítico em aplicações sujeitas a ambientes adversos, como aquelas expostas à radiação. Nesse cenário, técnicas de tolerância a falhas tornam-se essenciais para mitigar os efeitos de erros transitórios e permanentes. Esta pesquisa explora quatro abordagens complementares: a redundância modular em hardware, utilizada para aumentar a robustez do sistema por meio da replicação de módulos; a reconfiguração parcial dinâmica, que permite restaurar regiões afetadas sem a necessidade de reinicializar todo o dispositivo; a injeção automatizada de falhas, aplicada como método sistemático para avaliar a resiliência das arquiteturas desenvolvidas; e o scrubbing, responsável por corrigir falhas acumuladas na memória de configuração em SRAM. A implementação proposta constituiu uma plataforma experimental relevante para a investigação integrada dessas técnicas em FPGAs da família Zynq 7000 da AMD/Xilinx, possibilitando a obtenção de dados empíricos sobre o desempenho e a aplicabilidade de estratégias complementares de confiabilidade. Os resultados experimentais evidenciam a efetividade das soluções adotadas na preservação da integridade funcional sob condições adversas emuladas por meio da injeção de falhas, reforçando o potencial de estratégias dinâmicas de tolerância a falhas para viabilizar o uso de FPGAs COTS baseados em SRAM em aplicações críticas que demandam alta confiabilidade.