A confiabilidade de sistemas digitais baseados em \acrshortpl{fpga} é um fator crítico em aplicações expostas a ambientes severos, especialmente aquelas sujeitas à radiação. Nesse contexto, técnicas de tolerância a falhas são essenciais para mitigar os efeitos de falhas transitórias e permanentes. Esta pesquisa investiga quatro estratégias complementares: \textbf{(i)} redundância modular em hardware, que aumenta a robustez por meio da replicação de módulos; \textbf{(ii)} Reconfiguração PArcial Dinâmica, que permite a recuperação de regiões defeituosas sem a necessidade de reconfigurar todo o dispositivo; \textbf{(iii)} Injeção Automatizada de Falhas, empregada como método sistemático para avaliar a resiliência das arquiteturas propostas; e \textbf{(iv)} \acrlong{ms}, responsável pela correção de erros acumulados na memória de configuração do \acrshort{fpga} baseada em \acrshort{sram}. 

A implementação proposta estabelece uma plataforma experimental para a avaliação integrada dessas técnicas em dispositivos AMD/Xilinx Zynq-7000 \acrshortpl{fpga}, fornecendo dados empíricos sobre seu desempenho e aplicabilidade prática. Os resultados demonstram que a Reconfiguração Parcial reduz o tempo de reconfiguração em até 97\% em comparação à reprogramação completa do dispositivo, enquanto o scrubbing corrigiu com sucesso todas as inversões de um único bit durante as campanhas automatizadas de injeção de falhas. No entanto, inversões de múltiplos bit resultaram em perda funcional total após aproximadamente 3.500 injeções aleatórias, confirmando os limites da proteção \acrshort{secded}. A implementação de redundância tripla do processador PicoRV32I \acrshort{riscv} aumentou a utilização de lógica em 3,1$\times$, mas manteve o funcionamento correto sob condições de falha única. De modo geral, o uso combinado de redundância, scrubbing e Recnfiguração Parcial melhorou significativamente a resiliência do sistema, reforçando a viabilidade do emprego de \acrshortpl{fpga} comerciais baseados em memória \acrshort{sram} em aplicações críticas e espaciais.
