FPGAs baseados em SRAM são amplamente utilizados em aplicações de alto desempenho e missão crítica devido à sua flexibilidade e capacidade de reconfiguração. No entanto, sua vulnerabilidade a erros suaves — especialmente em ambientes hostis, como aeroespacial ou controle industrial — representa um desafio significativo à confiabilidade dos sistemas ao longo do tempo. Esta dissertação apresenta um estudo abrangente sobre a confiabilidade de arquiteturas baseadas em FPGA SRAM por meio de injeção de falhas experimental. Um circuito de referência, baseado em um núcleo de processador RISC-V modular, é implementado para avaliar o comportamento do sistema sob diferentes cenários de falhas injetadas. A pesquisa propõe uma arquitetura tolerante a falhas, avaliada por meio de campanhas de injeção de falhas em nível de simulação e em hardware. A detecção de falhas e a recuperação do sistema são exploradas utilizando uma abordagem de reconfiguração dinâmica em tempo de execução, permitindo a substituição seletiva de módulos sem a interrupção completa do sistema. A metodologia proposta permite uma avaliação quantitativa da cobertura de falhas, latência de recuperação e disponibilidade do sistema. Os resultados experimentais demonstram a eficácia das técnicas de mitigação na preservação da integridade funcional frente a condições adversas, validando o potencial de estratégias dinâmicas de confiabilidade para FPGAs baseados em SRAM.