SRAM-based Field-Programmable Gate Arrays (FPGAs) are widely adopted in safety-critical and high-performance applications due to their flexibility and reconfigurability. However, their susceptibility to soft errors—particularly in harsh environments such as aerospace or industrial control—poses a significant challenge to long-term system reliability. This dissertation presents a comprehensive study on the reliability of SRAM-based FPGA systems through experimental fault injection. A benchmark circuit, based on a modular RISC-V processor core, is implemented to evaluate fault behavior under different injection scenarios. The study introduces a fault-tolerant architecture that is evaluated using both simulation-level and hardware-level fault injection campaigns. Fault detection and system recovery are investigated through a runtime reconfiguration approach, enabling selective module repair without full system interruption. The proposed methodology allows for quantitative assessment of fault coverage, recovery latency, and system availability. Experimental results demonstrate the effectiveness of fault mitigation techniques in maintaining functional integrity under injected fault conditions, validating the potential of dynamic reliability strategies in SRAM-based FPGAs.
