The reliability of digital systems based on \acrshortpl{fpga} is a critical factor in applications exposed to harsh environments, particularly those affected by radiation. In this context, fault-tolerance techniques are essential to mitigate the effects of transient and permanent faults. This research investigates four complementary strategies: \textbf{(i)} hardware modular redundancy, which enhances robustness through module replication; \textbf{(ii)} \acrlong{dpr}, enabling the recovery of faulty regions without reprogramming the entire device; \textbf{(iii)} automated \acrlong{fi}, used as a systematic approach to assess the resilience of the proposed architectures; and \textbf{(iv)} \acrlong{ms}, responsible for correcting accumulated errors in the \acrshort{sram}-based \acrshort{cram}. 

The proposed implementation establishes an experimental platform for the integrated evaluation of these techniques on AMD/Xilinx Zynq-7000 \acrshortpl{fpga}, providing empirical data on their performance and practical applicability. The results demonstrate that \acrshort{dpr} reduces the reconfiguration time by up to 97\% compared to complete device reprogramming, while scrubbing successfully corrected all single-bit upsets during automated \acrshort{fi} campaigns. However, multi-bit upsets led to total functional loss after approximately 3,500 random injections, confirming the limits of \acrshort{secded} protection. The \acrshort{tmr} implementation of the PicoRV32 \acrshort{riscv} processor increased the utilization of logic by 3.1$\times$ but maintained correct operation under single-fault conditions. In general, the combined use of redundancy, scrubbing, and \acrshort{dpr} significantly improved system resilience, reinforcing the feasibility of deploying \acrshort{cots} \acrshort{sram}-based \acrshortpl{fpga} in harsh conditions, particularly in non-critical instruments and communication systems for space-grade applications.
