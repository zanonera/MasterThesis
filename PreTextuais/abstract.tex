The reliability of digital systems based on FPGAs is a critical aspect in applications exposed to harsh environments, such as those subject to radiation. In this context, fault-tolerance techniques are essential to mitigate the effects of transient and permanent errors. This research explores four complementary approaches: hardware modular redundancy, employed to enhance system robustness through module replication; dynamic partial reconfiguration, which allows restoring affected regions without the need to reset the entire device; automated fault injection, applied as a systematic method to evaluate the resilience of the developed architectures; and scrubbing, responsible for correcting accumulated errors in the SRAM-based configuration memory. The proposed implementation constitutes a relevant experimental platform for the integrated investigation of these techniques in AMD/Xilinx Zynq 7000 FPGAs, enabling the collection of empirical data on the performance and applicability of complementary reliability strategies. Experimental results demonstrate the effectiveness of the adopted solutions in preserving functional integrity under adverse conditions emulated through automated fault injection, reinforcing the potential of dynamic fault-tolerance strategies to enable the use of SRAM-based COTS FPGAs in critical applications that require high reliability.
