%\section{Concepts and Literature Review}
\label{sec:cap2}

This chapter presents the essential concepts and summarizes the state-of-the-art in FPGA reliability, fault-tolerant architectures, and error mitigation strategies. It provides the foundation upon which this dissertation builds, highlighting key technologies, methods, and prior contributions in the field.

\section{\glsentrytext{fpga}}
\glspl{fpga} are semiconductor devices that can be configured by the user after manufacturing to implement custom digital circuits. Their programmable nature enables rapid prototyping, hardware acceleration, and long-term design flexibility. This section introduces FPGA architectures, focusing on the role of configurable logic blocks, routing resources, and configuration memory.

\begin{figure}[ht]
\centering
\includegraphics[width=0.75\textwidth]{Cap2/fpga-structure.png}
\caption{FPGA Structure. Adapted from \cite{Marioli2010}}\label{fpga-arch}
\end{figure}

\section{Sources of Faults in \glsentrytext{fpga}}
FPGA devices are susceptible to faults that arise from various sources, such as radiation-induced soft errors, aging effects, manufacturing defects, and electromagnetic interference. In particular, SRAM-based FPGAs are vulnerable to \glspl{seu} that can corrupt configuration bits or logic states. This section classifies faults into transient, intermittent, and permanent categories, discussing their mechanisms and impact on system behavior.

\section{Fault-Tolerance and Mitigation Techniques}
Numerous methods have been proposed to enhance the resilience of \gls{fpga} systems. Redundancy-based approaches such as \gls{tmr} and \gls{nmr}, memory scrubbing, and configuration repair are among the most widely used. This section surveys existing techniques, comparing their error coverage, resource overhead, and suitability for different applications.

\section{\glsentrytext{dpr} in Fault Mitigation}
\gls{dpr} has emerged as a powerful tool for on-the-fly repair of faulty \gls{fpga} regions without interrupting the operation of the entire system. Previous works have demonstrated its use in self-healing architectures and adaptive systems. This section reviews these contributions, analyzing their advantages and limitations.

\section{Fault Injection as an Evaluation Methodology}
Fault injection is widely employed to test and validate fault-tolerant \gls{fpga} designs. By introducing controlled errors in either simulation or hardware, researchers can characterize system behavior under fault conditions. This section examines different fault injection techniques, such as software-based, emulation-based, and on-chip implementations, as well as their relevance for reliability assessment.

\section{Summary of Related Works}
This section consolidates previous research in the field of \gls{fpga} reliability enhancement, highlighting key trends, open challenges, and gaps that motivate this dissertation. A comparative analysis of state-of-the-art methods provides the basis for defining the novel contributions of this work.

\begin{table}[htbp]
\centering
\caption{Comparison of \gls{fpga} fault-tolerance approaches in related works}
\label{tab:related_works}
\begin{tabular}{|p{3cm}|p{2.5cm}|p{3cm}|p{3cm}|p{3cm}|}
\hline
\textbf{Reference / Approach} 
& \textbf{Fault-Tolerance Method} 
& \textbf{Evaluation Method} 
& \textbf{Advantages} 
& \textbf{Limitations} \\
\hline
Smith et al. (2021) 
& Triple Modular Redundancy (TMR) 
& Simulation-based fault injection 
& Simple design, high coverage 
& High area and power overhead \\
\hline
Garcia et al. (2020) 
& Dynamic Partial Reconfiguration (DPR) repair 
& On-chip fault injection 
& Real-time reconfiguration, reduced downtime 
& Complex controller, reconfiguration latency \\
\hline
Lee et al. (2019) 
& N-Modular Redundancy (NMR) voter circuits 
& Emulation with fault injection 
& Scalable reliability, configurable protection 
& Significant resource usage for large N \\
\hline
\textbf{This Work} 
& NMR with DPR and integrated fault injection 
& On-chip + software-in-the-loop evaluation 
& Low downtime, tunable redundancy, adaptive repair 
& Slightly higher design complexity \\
\hline
\end{tabular}
\end{table}