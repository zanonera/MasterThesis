%\section{Development Methodology}
\label{sec:cap4}

This chapter details the methodology adopted to design, implement, and validate the proposed fault-tolerance approach. It explains the workflow, design tools, experimental setup, and evaluation strategies applied throughout the development.

For easier development and debugging, the entire design was broken down into modules and developed independently before integrating it all together into one final architecture. For this, each mitigation was developed and tested stand alone, and ultimately all mitigations were implemented in a final solution with a \gls{tmr} \gls{riscv} 32I.

\section{FPGA Platform and Toolchain}
This section presents the selected FPGA device, development board, and software toolchain. The hardware platform was chosen considering resource availability, partial dynamic reconfiguration capabilities, and compatibility with on-chip fault injection tools. The toolchain description includes synthesis tools, placement-and-routing strategies, and configuration methods.

\begin{table}[ht]
\centering
\begin{adjustbox}{max width=\paperwidth}
\begin{tabular}{|l|p{11.5cm}|}
\hline
\textbf{Tool} & \textbf{Description} \\
\hline
\href{https://www.xilinx.com/products/design-tools/vivado.html}{Vivado 2018.3} &
Xilinx FPGA design suite used for HDL synthesis, implementation, and bitstream generation. Provides Tcl scripting support for automation. \\
\hline
\href{https://www.python.org/}{Python} &
High-level programming language for automation, scripting, and testbench development. Widely used to coordinate FPGA workflows and fault injection experiments. \\
\hline
\href{https://github.com/riscv-collab/riscv-gnu-toolchain}{RISC-V Toolchain} &
Open-source GNU toolchain supporting the RISC-V ISA, enabling firmware compilation and debugging for soft-core processors like PicoRV32 \cite{PICORV32I}. \\
\hline
\href{https://github.com/YosysHQ/picorv32/}{PicoRV32} &
Compact and configurable RISC-V soft processor core optimized for FPGA, often used for embedded control, monitoring, and fault injection control logic. \\
\hline
\end{tabular}
\end{adjustbox}
\caption{Software development tools used in this work.}
\label{tab:sw-tools}
\end{table}

All software resources applied in this work are presented in Table \ref{tab:sw-tools}. The development environment was based on the \href{https://releases.ubuntu.com/18.04/} {Linux Ubuntu 18.04} operating system, which provided stability and compatibility with the required tools.

\begin{table}[ht]
\centering
\begin{adjustbox}{max width=\paperwidth}
\begin{tabular}{|l|p{13.5cm}|}
\hline
\textbf{Tool} & \textbf{Description} \\
\hline
\href{https://www.amd.com/en/corporate/university-program/aup-boards/pynq-z2.html}{Pynq Z2} &
The PYNQ-Z2 is a development board designed for experimenting with FPGAs and embedded systems in an accessible way. It is based on the Xilinx Zynq-7000 SoC (System on Chip). Specifically the Zynq XC7Z020, the same device used in the famous \href{https://www.amd.com/en/search/partner/embedded-partner-solutions.html#/6821}{Zedboard}. \\
\hline
\href{https://ftdichip.com/products/ft2232hq/}{FT2232} &
The FT2232 is a USB-to-multi-purpose communication interface IC developed by FTDI (Future Technology Devices International).
It is widely used for bridging USB connections to various serial and parallel protocols, making it especially popular in FPGA development boards, embedded systems, and debugging tools. \\
\hline
\end{tabular}
\end{adjustbox}
\caption{Hardware development tools used in this work.}
\label{tab:hw-tools}
\end{table}

\section{Design Workflow}
The design workflow defines the sequence of steps taken to develop the hardware and software components. It includes the specification of system requirements, architecture selection, hardware description, simulation, and synthesis flow. The workflow was structured to ensure iterative refinement and early detection of design flaws before FPGA implementation.

In order to evaluate all necessary knowledge to implement the final solution, the first step was to develop an initial architecture with the needed components.

\begin{figure}[ht]
    \centering
    \includegraphics[width=1\textwidth]{Cap4/architecture.png}
    \caption{Block diagram representation of the proposed platform.}
    \label{zynq-paper-arch}
\end{figure}

As shown in Figure~\ref{zynq-paper-arch}, the core of the architecture is the Xilinx Zynq-7000, a \gls{soc} integrating a dual-core ARM A9 processor with an \gls{fpga} fabric on the same die. This architecture was used as the base design to implement \gls{dpr} and to mitigate errors through memory scrubbing.

These capabilities make the platform suitable for deployment in harsh environments, particularly for non-critical instrumentation and communication systems, as \gls{sram}-based \glspl{fpga} provide a cost- and area-efficient solution \cite{Brosser2014}.

\section{Implementation of N-Modular Redundancy}
Here, the integration of the NMR scheme is detailed. The voter architecture, module replication strategy, and resource allocation are discussed. Specific design trade-offs such as latency, area overhead, and power consumption are analyzed to justify the chosen parameters for $N>3$.

\begin{landscape}
    \begin{figure}[ht]
    \centering
    \includegraphics[width=1.5\textwidth]{Cap4/tmr-vivado.png}
    \caption{Block Diagram of the implementation in Vivado. In the blue square (left) the memory for the RISC-V instructions (firmware), in the red square (center) the triplicated RISC-V cores and in the black square (right) the word voters inserted in each output of the PicoRV32 RISC-V core.}\label{vivado-tmr}
\end{figure}\end{landscape}

\section{Fault Injection Strategy}
This subsection describes how faults were emulated in the FPGA to validate system robustness. Both hardware-based and software-assisted fault injection approaches are explained, along with fault models (single-event upsets, multiple-bit upsets) and their mapping to the FPGA fabric. The methodology ensures repeatability and coverage of critical failure modes.

\section{Partial Dynamic Reconfiguration Flow}
The implementation of partial dynamic reconfiguration is explained, focusing on how faulty modules can be repaired in real time without halting system operation. This includes the generation of partial bitstreams, the reconfiguration controller design, and performance metrics such as reconfiguration latency and system recovery time.

\gls{dpr} flow is supported by Vivado, but it is not yet fully integrated with the GUI-based project flow~\cite{vipin2018dpr}. 
The designs are implemented using the Vivado Tcl-based command flow or a combination of Tcl commands and the GUI. 
To facilitate the reproduction of the results and provide a base configuration for other studies using this architecture, a set of scripts was developed to create a project with two PR modules, implementing the following steps:

\begin{enumerate}
    \item Create the project and block design.
    \begin{enumerate}
        \item The block design should contain the ARM processing system instance.
        \item The reconfigurable items should have only the interface definitions declared in this step.
    \end{enumerate}
    \item Synthesize the design, generating all the design checkpoints (DCP) for the static region and blocks of logic.
    \item Load the DCP for the static region and one of the reconfigurable partitions (RP).
    \item Define RP properties.
    \item Run Design Rule Checker (DRC).
    \item Create and implement the first configuration.
    \item Create all other configurations.
    \item Verify all configurations using the \texttt{PR\_verify} command.
    \item Generate bitstreams.
\end{enumerate}

The \gls{dpr} process is carried out through the ARM \gls{pcap} interface. Initially, the full or partial bitstream is stored in the processor’s DDR memory, after which a DMA transfer sends it to the configuration memory of the programmable logic (PL) via the \gls{pcap} port.

\section{Integration with Processing System (PS)}
This section discusses how the ARM processing system in the SoC was used to monitor and control the fault-tolerant logic. Mitigation strategies were partially moved to the PS side to balance workload and reduce logic overhead in the programmable fabric.

\section{Verification and Validation Methodology}
The testing strategy is presented to demonstrate how the design was verified at simulation level, emulation level, and on the physical FPGA hardware. Metrics such as fault coverage, error latency, and resource usage were systematically measured to evaluate the effectiveness of the proposed solution.

\section{Summary}
This section summarizes the methodology chapter, highlighting the major design decisions and how they lay the groundwork for the results presented in the next chapter.
