%\section{Results and Analysis}
\label{sec:cap5}

This chapter presents the experimental results obtained from the proposed methodology and analyzes their significance. The discussion includes performance evaluation, fault-injection outcomes, comparison with reference approaches, and an assessment of the reliability improvements achieved.

\section{Experimental Setup}
This section describes the hardware platform, software environment, test configurations, and performance metrics used for the experiments. Details such as FPGA device characteristics, fault-injection parameters, and evaluation benchmarks are provided to ensure reproducibility.

\section{Baseline Performance Evaluation}
Before introducing fault-tolerance mechanisms, the system performance is evaluated under normal conditions. This subsection discusses metrics such as resource utilization, operating frequency, and execution latency to establish a baseline reference.

\section{Fault-Injection Campaign Results}
This section presents the results of the systematic fault-injection experiments. The effects of single-event upsets (SEUs) on configuration memory and logic are analyzed, highlighting error rates, propagation patterns, and observed system-level failures.

\section{Impact of Mitigation Techniques}
The results obtained after implementing the proposed fault-mitigation strategies are discussed. Comparisons are made between unprotected and protected designs, focusing on improvements in fault coverage, error recovery, and reliability metrics.

\section{Comparative Analysis with Existing Approaches}
This section benchmarks the proposed solution against similar works in the literature. A summary table highlights differences in methodology, fault coverage, resource overhead, and performance impact.

\section{Discussion of Findings}
The key findings are synthesized and interpreted in terms of practical implications. The trade-offs between reliability, performance, and resource consumption are critically assessed, providing insights for future improvements.

